%---------- Inleiding ---------------------------------------------------------

% TODO: Is dit voorstel gebaseerd op een paper van Research Methods die je
% vorig jaar hebt ingediend? Heb je daarbij eventueel samengewerkt met een
% andere student?
% Zo ja, haal dan de tekst hieronder uit commentaar en pas aan.

\paragraph{Opmerking}

Dit voorstel is gebaseerd op het onderzoeksvoorstel dat werd geschreven in het
kader van het vak Research Methods dat ik (vorig/dit) academiejaar heb
uitgewerkt (met medesturent Elias Van Meerssche als mede-auteur).


\section{Inleiding}
\label{sec:inleiding}

TESTLIJN

Binnen de HOGENT wordt er voor lessen met verplichte aanwezigheid gecontroleerd of studenten wel degelijk zijn komen opdagen. De manier waarop deze aanwezigheid word bijgehouden is door middel van een QR-code die geprojecteerd wordt. De student scant deze code wat leidt naar het schoolplatform (Chamilo) en bevestigt zo via zijn account dat hij fysiek aanwezig was in deze les. Deze methode geeft een duidelijk overzicht van de aanwezige studenten en daarmee ook wie er niet verschenen is. Het probleem bij deze methode van aanwezigheden afnemen is dat studenten de QR-code kunnen fotograferen en doorsturen naar hun medestudenten die al dan niet thuis zijn of ergens volledig anders. 

Het doel van deze bachelorproef is een antwoord vinden op de vraag: `Hoe kan geofencing worden toegepast om te helpen de aanwezigheidsfraude tegen te gaan binnen de verplichte lessen van HOGENT?' Door middel van een onderzoek naar de toepassingen van geofencing alsook de noden van een nieuw systeem zal dit worden uitgewerkt. Het uitwerken van dit onderzoek zal worden ondersteund door enkele deelvragen:
\begin{itemize}
  \item Wat moet deze uitbreiding op het huidig systeem bereiken?
  \item Hoe nauwkeurig zou het systeem moeten zijn en tot in welke mate beïnvloed dit de nuttigheid?
  \item Welke alternatieven op geofencing zijn er?
  \item In welke mate zal er iets veranderen langs de student hun kant en kan dit beperkt worden?
\end{itemize}

% probleemdomein: "Het uiteindelijke doel is een uitbreiding op het huidig systeem te implementeren die het aantal vervalsingen van aanwezigheid kan verlagen" -> wat is het probleem dat je probeerd op te lossen?

% probleemdomein: "De toepassing van geofencing is niet de enige oplossing voor dit scenario dus mogelijke alternatieven zullen besproken worden maar zijn hier niet het doel om uit te werken" -> zijn er alternatieven en waarom zou je deze wel/niet uitwerken?



% oplossingsdomein: "De nauwkeurigheid van deze uitbreiding zal ook deels bepalen hoe nuttig dit is" -> tot in hoever zal de accuraatheid uitmaken voor de nutigheid van de gekozen oplossing?

% oplossingsdomein: "Indien mogelijk wordt er gezocht naar een oplossing die zo min mogelijk verandering langs de student hun kant vereist voor gemak alsook privacy." -> tot in hoever kan de oplossing worden geimplementeerd zonder dat dit teveel stoort met de studenten/extra werk vereist van hun?


%---------- Stand van zaken ---------------------------------------------------

\section{Literatuurstudie}%
\label{sec:literatuurstudie}

%NOG WEGHALEN 

% DEELVRAGEN:

% - wat is het huidig systeem en wat zijn de problemen ervan?

Tijdens de lessen wordt een QR-code geprojecteerd vooraan op het bord, studenten krijgen vervolgens de tijd om deze code te scannen en zo worden ze aangemeld. Het huidig systeem heeft een paar problemen die het frauderen makelijk maken.  De QR-code blijft lang zichtbaar staan omdat er een 100 tal studenten deze moeten scannen. Naast een tijdelijke vertraging op Chamilo door alle studenten die zich tegelijk aanmelden, zorgt dit ook dat de code lang genoeg geprojecteerd moet blijven. In de tijd dat deze code zichtbaar staat nemen de studenten een foto om door te sturen naar hun medestudenten. Het is niet mogelijk hier te controlleren welke student dit doet en welke niet, aangezien het aantal studenten te hoog is om snel te controlleren. Eens aangemeld is het ook niet mogelijk om te weten of deze student degelijk aanwezig was op het moment van de les of niet.




% - wat zijn factoren dat het nieuwe systeem aan moet voldoen?
% - gdpr/privacy omtrent studenten p1
% - wat zijn mogelijke alternatieven of uitbreidingen?
% - hoe zou geofencing toepasbaar zijn op dit probleem?
Geofencing is niet de enige optie om het aanwezigheidsprobleem op te lossen. Er bestaan vele methodes die mogelijkheden bieden om te helpen bij dit probleem zoals biometrische authenticatie en RFID. Beiden hebben hun voordelen wat ze een mogelijk alternatief maakt voor geofencing maar niet zonder nadelen. Een belangrijke factor om mee rekening te houden is de GDPR (General Data Protection Regulation). De GDPR-wetgeving houdt onder andere in of de privacy van studenten gerespecteerd wordt bij het verzamelen van data. De twee vermelde voorbeelden worden hieronder kort besproken om te schetsen waarom ze zouden kunnen werken in dit onderzoek alsook waarom er beslist wordt op dit moment om hier niet mee verder te werken.

\subsection{Biometrische data}
Biometrische data wordt beschreven door \textcite{Jeunen2025} als een unieke set aan informatie die vergaard wordt via technische proccessen gelinkt aan een persoon zijn fysieke, mentale of 
gedrags-attributen. Dit kan onder andere gebeuren via gezichtsherkenning of vingerafrukken. Een voorbeeld van hoe dit kan worden toegepast is dat er gebruikt zou gemaakt worden van gezichtsherkenning door camera's in de lokalen. Alhoewel een camera op zichzelf is toegestaan, komt de GDPR pas ten sprake eens je deze recordings zou combineren met gezichtsherkennende software. Volgens \textcite{Persoonsgegevens2025} valt software gebruiken om gezichten te kunnen onderschijden van elkaar met als doel iemand te identificeren onder de criteria van biometrische data. Alle data die voldoet aan deze lijst moet dus ook voldoen aan GDPR-wetgevingen, anders is dit een misbruik van informatie. In 2019 had een Zweedse school geprobeerd te werken met aanwezigheden afnemen via gezichtsherkenning, maar er werd tijdens de proef periode vastgesteld door de Zweedse Data Protection Agreement (DPA) dat deze test moest stopgezet worden. De privacy van studenten werd niet gerespecteerd door deze constante monitoring en schende daarbij enkele regels van de GDPR. De verantwoordelijke school moest hiervoor 200.000 SEK betalen in schadevergoeding wat overeenkomt met 20.000 euro \autocite{Board2025}.

\subsection{Radio-Frequency IDentification (RFID)}
RFID is een technologie waarbij er via radio signalen kan data overgeplaatst worden zonder direct contact \autocite{Andrew}. Deze technologie wordt onder andere toegepast bij de studentenkaarten van HOGENT. Elke student bezit een kaart met zijn studenten foto, een barcode als identificatie voor de kaart maar ook een inwendige chip. Deze chip toont aan dat dit een geldige studentenkaart is want hij kan onder andere gebruikt worden om mee te betalen op campus. Deze technologie is heel flexibel en kan dus ook toegepast worden om aanwezigheden af te nemen. Bedrijven zoals Cykeo en Viaonda hebben zich gericht naar het ontwikkelen van een systeem dat toelaat aanwezigheden af te nemen via verschillende RFID-scanners buiten lokalen, aan de school ingang of in de hallen \autocites{ViaOnda2025}{Cykeo2025}. Deze methode van aanmelden is efficïent en al deels geïmplementeerd dankzij de studentenkaarten, maar om het volledig te kunnen benutten zou nog de installatie nodig zijn van de scanners doorheen de campus. Een van de voornaamste nadelen wel van deze methode voor aanwezigheden af te nemen is dat een student meerdere kaarten kan meenemen en scannen wat zorgt dat fraude mogelijk blijft.
% - wat is geofencing en zijn er variaties van?
\subsection{Geofencing}
Geofencing is een locatie-gebasseerde technologie die virtuele grenzen trekt rond een omgeving \autocite{Hoffman2022}. Deze toepassing kan op verschillende manieren gebruikt worden, gaande van gewoon informatie verzamelen hoe vaak mensen een plek passeren tot actief iemand zijn toegang tot applicaties of andere software beperken bij het verlaten van deze grens. Binnen deze methode kan een onderschijd gemaakt woden tussen twee types van geofencing, actieve en passieve:

\subsubsection{Actieve}
Actieve geofencing gebeurt doormiddel van de gebruiker zijn GPS-signaal om zo nauwkeurig de locatie te kunnen volgen en data op te halen. Deze methode is intensiever voor batterijgebruik door de constante data die doorgegeven wordt en de applicatie die blijft draaien. Een voordeel van deze methode is onder anderen dat er een melding of andere reactie kan geactiveerd worden indien de gebruiker een virtuele grens oversteekt of zich op een bepaalde locatie bevindt. Een voorbeeld hiervan is het spel `pokemon-go' waar de speler zijn locatie in het spel meebeweegd met zijn locatie in het echt \autocite{Hoffman2022}.

\subsubsection{Passieve}
Passieve geofencing is in tegenstelling tot actieve geofencing niet intensief op de baterij, noch heeft het GPS verbinding nodig. Deze variant werkt doormiddel van wifi, zendmasten of gelijkaardige methodes. Indien een gsm verbindt met een van deze opties, kan zijn positie grofweg worden vastgesteld omdat de gebruiker zich in de buurt van het signaal moet bevinden. Op basis hiervan kunnen bedrijven signalen geven naar consumenten zoals reclame of korting \autocite{Hoffman2022}. Een voorbeeld van deze methode is Proximus of Telenet die een bericht sturen als je naar het buitenland gaat omdat je verbindt met een zendmast voorbij de grens.

% - gdpr/privacy omtrent studenten p2
Voor de uitwerking van deze bachelorproef zal er eerder gekozen worden voor passieve geofencing. Deze methode vereist minder input van de studenten zoals hun locatie zichtbaar zetten en is minder intensief voor hun gsm-batterij. Bij het uitwerken van hoe deze passieve geofencing kan geïmplementeerd worden, zal moeten opgepast worden dat hier geen schending gebeurt van de GDPR-wetgeving aangezien dit, zoals eerder vermeld kan leiden tot sancties gaande van het stopzetten van het project tot financiële vergoedingen.

% - wat is er nodig om het alternatief te implementeren?


% - wat zal het verschil zijn tussen het hudig systeem en het voorgestelde alternatief?

% Je mag deze sectie nog verder onderverdelen in subsecties als dit de structuur van de tekst kan verduidelijken.

%---------- Methodologie ------------------------------------------------------
\section{Methodologie}
\label{sec:methodologie} 

\subsection{Probleemanalyse (3 weken)}
% (NEEM GROFWEG 14 WEKEN en pas de tijden aan)
De eerste fase van deze bachelorproef draait rond meer informatie verzamelen om het probleem te verduidelijken alsook een voorspelling op te stellen van mogelijke struikelblokken. Doormiddel van enkele studenten en leerkrachten te ondervragen wordt onderzocht of er nog methodes van fraude zijn die nog niet vermeld werden alsook suggesties van hoe dit tegen te gaan. Naast deze ondervraging wordt ook opgesteld welke andere onderdelen van het onderzoek een invloed zullen hebben op de uiteindelijke uitwerking. Het gaat hier over de technologie die ter beschikking staat, de nodige kennis voor het uitvoeren van de test en wettelijke restricties zoals privacy. Dit zou gebundeld worden tot een overzicht dat gebruikt wordt bij het verder uitwerken van de bachelorproef om te helpen problemen hierrond te vermijden.


\subsection{Oplossing ontwikkeling (4 weken)}
In het tweede deel van het onderzoek zal uitgezocht worden wat nu exact de uitwerking wordt voor op de testopstelling. Aangezien op dit moment nog geen testopstelling vast staat kunnen de noden hier veranderen afhankelijk van wat er uitgewerkt wordt. De mogelijke uitwerking zal al eens gecontroleerd worden met het overzicht van de vorige stap om kansen op problemen te verlagen. Het resultaat van deze stap zal een concreet plan zijn van hoe de oplossing ontwikkeld moet worden en  wat deze inhoudt.


\subsection{Proof-of-concept (6 weken)}
Het derde deel van de bachelorproef zal het uitwerken van de proof-of-concept zijn. Hier wordt de kennis en informatie van de vorige stappen gerealiseerd in een geheel. Als eerste stap van deze uitvoering, zal de testomgeving worden opgesteld. Deze opstelling zal gebruikt worden om te kunnen testen dat de uitgewerkte oplossing wel degelijk functioneel is. Vervolgens zal op deze opstelling de nodige handelingen worden uitgevoerd om de uitgewerkte aanmeldingsmethode te kunnen testen. Deze testen zullen verschillende delen van het onderzoek staven zoals hoe ver de gemeten afstand is vergeleken de werkelijke locatie of tot in hoever een student zijn locatie niet ophaalbaar is.


\subsection{Evaluatie (1 week)}
Het laatste deel van het onderzoek is het noteren en uitwerken van de behaalde resultaten. De resultaten van de proof-of-concept zullen worden genoteerd zodat achteraf kan gekeken worden hoe het onderzoek verliep. Er zal bij deze evaluatie genoteerd worden of er nog onverwachte problemen opdoken, of de nauwkeurigheid van de geofencing afweek van wat er verwacht werd en hoe het testen verliep. Het samenstellen en evalueren van deze resultaten zal ongeveer een week duren, en op basis hiervan zal een conclussie opgesteld worden van het gehele onderzoek.

%---------- Verwachte resultaten ----------------------------------------------
\section{Verwacht resultaat, conclusie}%
\label{sec:verwachte_resultaten}

Het verwacht resultaat van de opstelling is dat de opstelling erin zal slagen een test persoon zijn locatie te bepalen bij het scannen van een QR-code. Alhoewel de accuraatheid niet volledig gegarandeerd kan worden zal dit aantonen dat het onderzoek tot in zekere mate heeft gewerkt. Hoe accuraat dit moet zijn en of hier nog verdere uitbreidingen bij horen kan in een verder onderzoek worden uitgewerkt. Het andere mogelijke resultaat van het onderzoek is dat de uitwerking niet geslaagd is. Dit kan onder andere gebeurd zijn omdat er een onverwacht probleem opdook tijdens de proof-of-concept of omdat de nodige technologie niet beschikbaar was door licenties of ontbrekende kennis.