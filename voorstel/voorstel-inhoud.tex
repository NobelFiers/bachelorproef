%---------- Inleiding ---------------------------------------------------------

% TODO: Is dit voorstel gebaseerd op een paper van Research Methods die je
% vorig jaar hebt ingediend? Heb je daarbij eventueel samengewerkt met een
% andere student?
% Zo ja, haal dan de tekst hieronder uit commentaar en pas aan.

\paragraph{Opmerking}

Dit voorstel is gebaseerd op het onderzoeksvoorstel dat werd geschreven in het
kader van het vak Research Methods dat ik (vorig/dit) academiejaar heb
uitgewerkt (met medesturent Elias Van Meerssche als mede-auteur).


\section{Inleiding}
\label{sec:inleiding}

Binnen de HOGENT wordt er voor lessen met verplichte aanwezigheid gecontroleerd of studenten degelijk zijn komen opdagen. De manier waarop deze aanwezigheid word bijgehouden is doormiddel van een QR-code die geprojecteerd wordt. De student scanned deze code wat leid naar het schoolplatform (Chamilo) en bevestigd zo via zijn account dat hij fysiek aanwezig was in deze les. Deze methode geeft een duidelijk overzicht van de aanwezige studenten en daarmee ook wie er niet verschenen is. Het probleem bij deze methode van aanwezigheden afnemen is dat studenten de QR-code kunnen fotograferen en doorsturen naar hun medestudenten die al dan niet thuis zijn of ergens volledig anders. 

Het doel van deze bachelorproef is onderzoeken hoe geofencing kan worden toegepast om te helpen de aanwezigheids fraude tegen te gaan binnen de verplichte lessen van HOGENT. Doormiddel van een onderzoek naar de toepassingen van geofencing alsook de noden van een nieuw systeem zal dit worden uitgewerkt. Het uiteindelijke doel is een uitbreiding op het huidig systeem implementeren die het aantal vervalsingen van aanwezigheid kan verlagen. De toepassing van geofencing is niet de enigste oplossing voor dit scenario dus mogelijke alternatieven zullen besproken worden maar zijn hier niet het doel om uit te werken.


%---------- Stand van zaken ---------------------------------------------------

\section{Literatuurstudie}%
\label{sec:literatuurstudie}

%NOG WEGHALEN 

% DEELVRAGEN:

% - wat is het huidig systeem en wat zijn de problemen ervan?
Tijdens de lessen word een QR-code geprojecteerd vanvoor op het bord, studenten krijgen vervolgens de tijd om deze code te scannen en zo worden ze aangemeld. In de tijd dat deze code aan het bord staat kunnen studenten een foto trekken hiervan en doorsturen naar hun medestudenten overal. Het huidig systeem heeft een paar problemen die dit zo makelijk maken te doen. 
- Een QR-code altijd hetzelfde van functionaliteit, ongeacht of het de echte is of een foto ervan. 
- Er is ruime tijd om de code te scannen aangezien het relatief lang duurt om chamilo in te laden wanneer een 100 tal studenten tegelijk probeerd aan te melden.
- Er zijn teveel studenten om te controlleren of ze fotos aan het trekken zijn of degelijk de code aan het scannen zijn.
- De aanmelding doet niks meer dan bevestigen dat student X de QR-code heeft gescanned.
De combinatie van deze verschillende problemen maakt het huidig zo makelijk om aan aanwezigheidsfraude te plegen. 

% - wat zijn factoren dat het nieuwe systeem aan moet voldoen?
% - gdpr/privacy omtrent studenten p1
Bij het ontwikkelen van een alternatief moet er dus gezocht worden of er een paar van deze problemen kunnen worden vermeden. Een andere belangrijke factor die moet worden bijgehouden, is hoe ze liggen binnen de grenzen van de GDPR-wetgeving. De Europese wetgeving rond data veiligheid en privacy ook wel bekend als de General Data Protection Regulation (GDPR) is een van de belangerijkste factoren om mee rekening te houden tijdens dit onderzoek. Enkele nuttige methodes die besproken zullen worden waarop dit toepasbaar is, is biometrische data, RFID en GPS/locatie-gebasseerde data.
\subsection{biometrische data}
Biometrische data word beschreven door \textcite{9NINEID} als "Biometrische data is gedefinieerd door de GDPR"
Een voorbeeld van hoe dit kan worden toegepast is als er gebruikt zou gemaakt worden van gezichtsherkenning door cameras in de lokalen. Alhoewel een camera op zichzelf is toegestaan, komt de GDPR pas ten sprake eens je deze recordings zou combineren met gezichtsherkennende software. Volgens \textcite{Persoonsgegevens2025} valt software gebruiken om gezichten te analyseren en zo ze te kunnen onderschijden van elkaar met als doel iemand te identificeren onder de criteria van biometrische data. Alle data die voldoet aan deze lijst moet dus ook voldoen aan GDPR-wetgevingen, anders is dit een misbruik van informatie. 
  https://gdpr-info.eu/

% - wat zijn mogelijke alternatieven of uitbreidingen?


% - wat is geofencing en zijn er variaties van?


% - hoe zou geofencing toepasbaar zijn op dit probleem?


% - gdpr/privacy omtrent studenten p2


% - wat is er nodig om het alternatief te implementeren?


% - wat zal het verschil zijn tussen het hudig systeem en het voorgestelde alternatief?





Hier beschrijf je de \emph{state-of-the-art} rondom je gekozen onderzoeksdomein, d.w.z.\ een inleidende, doorlopende tekst over het onderzoeksdomein van je bachelorproef. Je steunt daarbij heel sterk op de professionele \emph{vakliteratuur}, en niet zozeer op populariserende teksten voor een breed publiek. Wat is de huidige stand van zaken in dit domein, en wat zijn nog eventuele open vragen (die misschien de aanleiding waren tot je onderzoeksvraag!)?

Je mag de titel van deze sectie ook aanpassen (literatuurstudie, stand van zaken, enz.). Zijn er al gelijkaardige onderzoeken gevoerd? Wat concluderen ze? Wat is het verschil met jouw onderzoek?

Verwijs bij elke introductie van een term of bewering over het domein naar de vakliteratuur, bijvoorbeeld~\autocite{Hykes2013}! Denk zeker goed na welke werken je refereert en waarom.

Draag zorg voor correcte literatuurverwijzingen! Een bronvermelding hoort thuis \emph{binnen} de zin waar je je op die bron baseert, dus niet er buiten! Maak meteen een verwijzing als je gebruik maakt van een bron. Doe dit dus \emph{niet} aan het einde van een lange paragraaf. Baseer nooit teveel aansluitende tekst op eenzelfde bron.

Als je informatie over bronnen verzamelt in JabRef, zorg er dan voor dat alle nodige info aanwezig is om de bron terug te vinden (zoals uitvoerig besproken in de lessen Research Methods).

% Voor literatuurverwijzingen zijn er twee belangrijke commando's:
% \autocite{KEY} => (Auteur, jaartal) Gebruik dit als de naam van de auteur
%   geen onderdeel is van de zin.
% \textcite{KEY} => Auteur (jaartal)  Gebruik dit als de auteursnaam wel een
%   functie heeft in de zin (bv. ``Uit onderzoek door Doll & Hill (1954) bleek
%   ...'')

Je mag deze sectie nog verder onderverdelen in subsecties als dit de structuur van de tekst kan verduidelijken.

%---------- Methodologie ------------------------------------------------------
\section{Methodologie}%
\label{sec:methodologie}

Hier beschrijf je hoe je van plan bent het onderzoek te voeren. Welke onderzoekstechniek ga je toepassen om elk van je onderzoeksvragen te beantwoorden? Gebruik je hiervoor literatuurstudie, interviews met belanghebbenden (bv.~voor requirements-analyse), experimenten, simulaties, vergelijkende studie, risico-analyse, PoC, \ldots?

Valt je onderwerp onder één van de typische soorten bachelorproeven die besproken zijn in de lessen Research Methods (bv.\ vergelijkende studie of risico-analyse)? Zorg er dan ook voor dat we duidelijk de verschillende stappen terug vinden die we verwachten in dit soort onderzoek!

Vermijd onderzoekstechnieken die geen objectieve, meetbare resultaten kunnen opleveren. Enquêtes, bijvoorbeeld, zijn voor een bachelorproef informatica meestal \textbf{niet geschikt}. De antwoorden zijn eerder meningen dan feiten en in de praktijk blijkt het ook bijzonder moeilijk om voldoende respondenten te vinden. Studenten die een enquête willen voeren, hebben meestal ook geen goede definitie van de populatie, waardoor ook niet kan aangetoond worden dat eventuele resultaten representatief zijn.

Uit dit onderdeel moet duidelijk naar voor komen dat je bachelorproef ook technisch voldoen\-de diepgang zal bevatten. Het zou niet kloppen als een bachelorproef informatica ook door bv.\ een student marketing zou kunnen uitgevoerd worden.

Je beschrijft ook al welke tools (hardware, software, diensten, \ldots) je denkt hiervoor te gebruiken of te ontwikkelen.

Probeer ook een tijdschatting te maken. Hoe lang zal je met elke fase van je onderzoek bezig zijn en wat zijn de concrete \emph{deliverables} in elke fase?

%---------- Verwachte resultaten ----------------------------------------------
\section{Verwacht resultaat, conclusie}%
\label{sec:verwachte_resultaten}

Hier beschrijf je welke resultaten je verwacht. Als je metingen en simulaties uitvoert, kan je hier al mock-ups maken van de grafieken samen met de verwachte conclusies. Benoem zeker al je assen en de onderdelen van de grafiek die je gaat gebruiken. Dit zorgt ervoor dat je concreet weet welk soort data je moet verzamelen en hoe je die moet meten.

Wat heeft de doelgroep van je onderzoek aan het resultaat? Op welke manier zorgt jouw bachelorproef voor een meerwaarde?

Hier beschrijf je wat je verwacht uit je onderzoek, met de motivatie waarom. Het is \textbf{niet} erg indien uit je onderzoek andere resultaten en conclusies vloeien dan dat je hier beschrijft: het is dan juist interessant om te onderzoeken waarom jouw hypothesen niet overeenkomen met de resultaten.

