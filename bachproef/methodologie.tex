%%=============================================================================
%% Methodologie
%%=============================================================================

\chapter{\IfLanguageName{dutch}{Methodologie}{Methodology}}%
\label{ch:methodologie}

\subsection{Probleemanalyse (3 weken)}
% (NEEM GROFWEG 14 WEKEN en pas de tijden aan)
De eerste fase van deze bachelorproef draait rond meer informatie verzamelen om het probleem te verduidelijken alsook een voorspelling op te stellen van mogelijke struikelblokken. Doormiddel van enkele studenten en leerkrachten te ondervragen wordt onderzocht of er nog methodes van fraude zijn die nog niet vermeld werden alsook suggesties van hoe dit tegen te gaan. Naast deze ondervraging wordt ook opgesteld welke andere onderdelen van het onderzoek een invloed zullen hebben op de uiteindelijke uitwerking. Het gaat hier over de technologie die ter beschikking staat, de nodige kennis voor het uitvoeren van de test en wettelijke restricties zoals privacy. Dit zou gebundeld worden tot een overzicht dat gebruikt wordt bij het verder uitwerken van de bachelorproef om te helpen problemen hierrond te vermijden.


\subsection{Oplossing ontwikkeling (4 weken)}
In het tweede deel van het onderzoek zal uitgezocht worden wat nu exact de uitwerking wordt voor op de testopstelling. Aangezien op dit moment nog geen testopstelling vast staat kunnen de noden hier veranderen afhankelijk van wat er uitgewerkt wordt. De mogelijke uitwerking zal al eens gecontroleerd worden met het overzicht van de vorige stap om kansen op problemen te verlagen. Het resultaat van deze stap zal een concreet plan zijn van hoe de oplossing ontwikkeld moet worden en  wat deze inhoudt.


\subsection{Proof-of-concept (6 weken)}
Het derde deel van de bachelorproef zal het uitwerken van de proof-of-concept zijn. Hier wordt de kennis en informatie van de vorige stappen gerealiseerd in een geheel. Als eerste stap van deze uitvoering, zal de testomgeving worden opgesteld. Deze opstelling zal gebruikt worden om te kunnen testen dat de uitgewerkte oplossing wel degelijk functioneel is. Vervolgens zal op deze opstelling de nodige handelingen worden uitgevoerd om de uitgewerkte aanmeldingsmethode te kunnen testen. Deze testen zullen verschillende delen van het onderzoek staven zoals hoe ver de gemeten afstand is vergeleken de werkelijke locatie of tot in hoever een student zijn locatie niet ophaalbaar is.


\subsection{Evaluatie (1 week)}
Het laatste deel van het onderzoek is het noteren en uitwerken van de behaalde resultaten. De resultaten van de proof-of-concept zullen worden genoteerd zodat achteraf kan gekeken worden hoe het onderzoek verliep. Er zal bij deze evaluatie genoteerd worden of er nog onverwachte problemen opdoken, of de nauwkeurigheid van de geofencing afweek van wat er verwacht werd en hoe het testen verliep. Het samenstellen en evalueren van deze resultaten zal ongeveer een week duren, en op basis hiervan zal een conclussie opgesteld worden van het gehele onderzoek.























%% TODO: In dit hoofstuk geef je een korte toelichting over hoe je te werk bent
%% gegaan. Verdeel je onderzoek in grote fasen, en licht in elke fase toe wat
%% de doelstelling was, welke deliverables daar uit gekomen zijn, en welke
%% onderzoeksmethoden je daarbij toegepast hebt. Verantwoord waarom je
%% op deze manier te werk gegaan bent.
%% 
%% Voorbeelden van zulke fasen zijn: literatuurstudie, opstellen van een
%% requirements-analyse, opstellen long-list (bij vergelijkende studie),
%% selectie van geschikte tools (bij vergelijkende studie, "short-list"),
%% opzetten testopstelling/PoC, uitvoeren testen en verzamelen
%% van resultaten, analyse van resultaten, ...
%%
%% !!!!! LET OP !!!!!
%%
%% Het is uitdrukkelijk NIET de bedoeling dat je het grootste deel van de corpus
%% van je bachelorproef in dit hoofstuk verwerkt! Dit hoofdstuk is eerder een
%% kort overzicht van je plan van aanpak.
%%
%% Maak voor elke fase (behalve het literatuuronderzoek) een NIEUW HOOFDSTUK aan
%% en geef het een gepaste titel.

\lipsum[21-25]

